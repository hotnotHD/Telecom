\subsection{Упражнение 1}

В этом упражнении требуется реализовать алгоритм Быстрого преобразования Фурье, его время работы пропорционально n logn

\begin{lstlisting}[language=Python]
import numpy as np
PI2 = 2 * np.pi
\end{lstlisting}

Возьмем простой массив

\begin{lstlisting}
ys = [0.2, 0.3, -0.7, -0.2]
hs = np.fft.fft(ys)
hs

array([-0.4+0.j ,  0.9-0.5j, -0.6+0.j ,  0.9+0.5j])
\end{lstlisting}

Дискретное преобразование Фурье

\begin{lstlisting}[language=Python]
def dft(ys):
    N = len(ys)
    ts = np.arange(N) / N
    freqs = np.arange(N)
    args = np.outer(ts, freqs)
    M = np.exp(1j * PI2 * args)
    amps = M.conj().transpose().dot(ys)
    return amps
\end{lstlisting}

\begin{lstlisting}[language=Python]
dft(ys)

array([-0.4+0.j ,  0.9-0.5j, -0.6-0.j ,  0.9+0.5j])
\end{lstlisting}

Функция для рекурсивного быстрого преобразования Фурье

\begin{lstlisting}[language=Python]
def fft_1(ys):
    N = len(ys)
    e_arr = np.fft.fft(ys[::2])
    o_arr = np.fft.fft(ys[1::2])

    ns = np.arange(N)
    W = np.exp(-1j * PI2 * ns / N)

    return np.tile(e_arr, 2) + W * np.tile(o_arr, 2)
    
fft_1(ys)

array([-0.4+0.j ,  0.9-0.5j, -0.6-0.j ,  0.9+0.5j])
\end{lstlisting}

Теперь применим рекурсию для np.fft.fft

\begin{lstlisting}[language=Python]
def fft_2(ys):
    if len(ys) == 1:
      return ys

    e_arr = fft_1(ys[::2])
    o_arr = fft_1(ys[1::2])

    ns = np.arange(len(ys))
    W = np.exp(-1j * PI2 * ns / len(ys))
    
    return np.tile(e_arr, 2) + W * np.tile(o_arr, 2)
    
fft_2(ys)

array([-0.4+0.j ,  0.9-0.5j, -0.6-0.j ,  0.9+0.5j])
\end{lstlisting}

\subsection{Вывод}

Дискретное преобразование Фурье  — это одно из преобразований Фурье, широко применяемых в алгоритмах цифровой обработки сигналов , а также в других областях, связанных с анализом частот в дискретномсигнале. Дискретное преобразование Фурье требует в качестве входа дискретную функцию. Такие функции часто создаются путём дискретизации. В качестве упражнения была написана одна из реализаций БПФ.